\section{Arithmetic Integer ABB Functionality}

The Arithmetic Integer ABB allows the inputting and outputting of elements of the ring $\Ztwol$, and basic operations on these elements.
Working over $\Ztwol$ is useful in many practical settings.
It allows for very efficient arithmetic, particularly additions, which can be implemented directly using native machine instructions.
Values in $\Ztwol$ also naturally correspond to $l$-bit integers, making this representation compatible with standard integer and bitwise data types used in computation.
In order to make explicit the distinction between operations in the ring $\Ztwol$, we denote addition and multiplication using $\Aadd$ and $\Amult$, respectively.
This is summarized in the functionality definition below.
More detailed descriptions, as well as the protocols implementing each of these functionalities, will be presented in later sections.

\begin{functionality}
	
	\InputArith{x}: Given a public integer $x$ in $\Ztwol$, store it in the ABB and return a public identifier to the variable. 
	(Shorthand: $\aint{x}$.)\\
	
	\InputArithFrom{x}{i}:  Given a private value, $\private{x}{i}$, held by $P_i$, where $x \in \Ztwol$,
	store $x$ in the ABB and return a public identifier to the variable. 
    Note that $P_i$'s local \emph{identifier} for $\private{x}{i}$ can also be private to $P_i$. \\
	
	\OutputArith{\aint{x}}: Given an element of $\Ztwol$ which is stored in the ABB under identifier $x$, reveal it to all parties. \\
	
	\OutputArithTo{\aint{x}}{i}:  Given an element of $\Ztwol$ which is stored in the ABB under identifier $x$, reveal it to $P_i$. \\
	
	\ArithAdd{\aint{x}}{\aint{y}}: Given $x$ and $y$, both elements of $\Ztwol$ stored in the ABB, return \aint{x \Aadd y}. (Shorthand: $\aint{x} \Aadd \aint{y}$.) \\

    \ArithMultConst{\aint{x}}{c}: Given $x$ and $c$, both elements of $\Ztwol$, where $x$ is stored in the ABB and $c$ is public, return $\aint{x \Amult c}$. \\

	\ArithtoBitArr{\aint{x}}: Given an element of $\Ztwol$, which is stored in the ABB under identifier $x$, create an array of bits representing $x$, and store this array in the Boolean Bit ABB. \\

	\BitArrtoArith{\bint{x}}: Given a value $x$ stored in the Boolean Integer ABB, store it in the ABB and return a public identifier to the variable. \\

\end{functionality}
