\section{Bit-Shift Operations}

Bit-shifts are useful operations which, in the Boolean share representation,
are very simple and efficient to implement.
There are two types of bit-shifts: left bit-shifts and right bit-shifts.
Both take as a parameter the number of places to shift, $s$.
A right bit-shift moves all bits to the right by $s$ places,
with the lowest-order $s$ bits being dropped. 
Since the value should remain an $l$-bit number, 
0s are placed in the $s$ highest-order locations.
A left bit-shift moves all bits $s$ places to the left, 
appending $s$ 0s as the new lowest-order bits
and dropping the $s$ highest-order bits.
A left bit-shift is equivalent to multiplying the number by $2^s$
modulo $2^l$,
while a right-shift is equivalent to dividing the number by
$2^s$, dropping any remainder.
Both protocols assume $l \geq s$.

\begin{protocol}[BitShift]
	$\LeftShift{\bint{x}}{s}$:\\
	\indent $\bitvec{X}$ = B2BitArr($\bint{x}$)\\
	\indent $l = $len($\bitvec{X}$) \\
	\indent for $i = 0, \ldots, l-s-1$:\\
	\indent \indent $\bitvec{Y_i} = \bitvec{X_{i+s}}$ \\
	\indent for $i = l-s, \ldots, l-1$:\\
	\indent \indent $\bitvec{Y_i} =$ InputBit(0) \\
	\indent $\bint{y} = $BitArr2Int($\bitvec{Y}$) \\
	\indent return $\bint{y}$ \\

\noindent $\RightShift{\bint{x}}{s}$:\\
	\indent $\bitvec{X}$ = B2BitArr($\bint{x}$)\\
	\indent $l = $len$(\bitvec{X}$) \\
	\indent for $i = 0, \ldots, s-1$: \\
	\indent \indent $\bitvec{Y_i} = $ InputBit(0) \\
	\indent for $i = s, \ldots, l-1$: \\
	\indent \indent $\bitvec{Y_i} = \bitvec{X_{i-s}}$ \\
	\indent $\bint{y} = $BitArr2Int($\bitvec{Y})$ \\
	\indent return $\bint{y}$
\end{protocol}



