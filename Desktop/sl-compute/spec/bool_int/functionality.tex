\section{Boolean Integer ABB Functionality}


The Boolean Integer ABB allows the inputting and outputting of integers, and basic
operations on the bitwise representations of these integers. Each integer is 
represented using a fixed number of bits $k$, which may vary across Boolean 
integers but is restricted to powers of $2$ for convenience. This is summarized in
the functionality definition below. More detailed descriptions, as well as the 
protocols implementing each of these functionalities, will be presented in later sections.

\begin{functionality}
	
	\InputInt{x}{k}: Given a public integer $x$, store it's boolean representation 
	of length $k$ in the ABB and return a public identifier to the variable. 
	(Shorthand: $\bint{x}$.)\\
	
	\InputIntFrom{x}{k}{i}:  Given a private integer, $\private{x}{i}$, held by $P_i$,
	store it's boolean representation of length $k$ in the ABB and return a public identifier 
	to the variable. Note that $P_i$'s local \emph{identifier} for $\private{x}{i}$ can 
	also be private to $P_i$. \\
	
	\OutputInt{\bint{x}}: Given an integer (it's boolean representation) stored in 
	the ABB under identifier $x$, reveal the integer to all parties. \\
	
	\OutputIntTo{\bint{x}}{i}:  Given an integer (it's boolean representation) stored 
	in the ABB under identifier $x$, reveal the integer to $P_i$.  \\
	
	\EqualizeLength{\bint{x}}{\bint{y}}:  Given two integers (their boolean 
	representations with lengths $k_x$ and $k_y$) stored 
	in the ABB under identifier $x$ and $y$, return the two integers stored 
	in the ABB such that the lengths of both their boolean representations is 
	extended to $\max(k_x, k_y)$.  \\
	
	\RandBoolInt{k}: Return $\bint{x}$, whose boolean representation is of length $k$,
	where $x \in_r \{0,\ldots,2^k-1\}$ with probability $\frac{1}{2^k}$. \\
	
	\InttoBitArr{\bint{x}}: Given an integer (it's boolean representation) stored in 
	the ABB under identifier $x$, return a vector of its boolean representation stored
	in the Boolean Bit ABB. \\

	\BitArrtoInt{\bitvec{X}}: Given a vector of boolean values stored in the Boolean Bit ABB, store 
	these boolean values in the ABB and return a public identifier to the variable. \\
	
	\Add{\bint{x}}{\bint{y}}: Return \bint{x + y}. (Infix: $\bint{x} + \bint{y}$.)\\
	
	\Sub{\bint{x}}{\bint{y}}: Return \bint{x - y}. (Infix: $\bint{x} - \bint{y}$.)\\
	
	\CompareEqual{\bint{x}}{\bint{y}}: If $x = y$, return $\bit{1}$. 
	Else return \bit{0}. (Infix: $\bint{x} == \bint{y}$.)\\
	
	\CompareGreaterThanOrEqualTo{\bint{x}}{\bint{y}}: If $x \ge y$, return $\bit{1}$. 
	Else return \bit{0}. (Infix: $\bint{x} \ge \bint{y}$.)\\
	
	\CompareLessThanOrEqualTo{\bint{x}}{\bint{y}}: If $x \le y$, return $\bit{1}$. 
	Else return \bit{0}. (Infix: $\bint{x} \le \bint{y}$.)\\
	
	\CompareGreaterThan{\bint{x}}{\bint{y}}: If $x > y$, return $\bit{1}$. 
	Else return \bit{0}. (Infix: $\bint{x} > \bint{y}$.)\\
	
	\CompareLessThan{\bint{x}}{\bint{y}}: If $x < y$, return $\bit{1}$. 
	Else return \bit{0}. (Infix: $\bint{x} < \bint{y}$.)\\
	
	\LeftShift{\bint{x}}{s}: Return \bint{(x \cdot 2^s) $ mod $ 2^k}. \\
	
	\RightShift{\bint{x}}{s}: Return \bint{(x / 2^s)}. \\
	
	\IfThenElse{\bit{c}}{\bint{x}}{\bint{y}}: If $c=1$ return $\bint{x}$, else return $\bint{y}$.
	
\end{functionality}
