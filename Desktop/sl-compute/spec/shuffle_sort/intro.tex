\section{Introduction}

Shuffling and sorting are core functions which are useful building blocks for
functions which analyze data.
Shuffling allows previous patterns in the arrangement of data to be removed,
which is a necessary step in many other protocols.
Sorting allows the accessing of data based on rank.

\begin{functionality}[Shuffle and Sort]
	\Shuffle{\bvec{X}}:
	\begin{enumerate}
		\item Let $n$ be the length of $\vec{X}$.
		\item Pick a random permutation 
			$\pi: \{0, \ldots, n-1\} \rightarrow \{0, \ldots, n-1\}$.
		\item For $0 \leq i \leq n-1$, set $\vec{Y}_i = \vec{X}_{\pi(i)}$. 
		\item Return $\bvec{Y}$.
	\end{enumerate}

	Sort...

\end{functionality}

Note that in \Shuffle{} the permutation used by the functionality is not revealed
to any of the parties, but is randomly chosen by the functionality.
This is different from the functionality \Perm{}{} presented in 
Protocol \ref{protocol:perm} in Section \ref{sec:preprocessing},
in which the PRG seed which defines the permutation is a public value.



