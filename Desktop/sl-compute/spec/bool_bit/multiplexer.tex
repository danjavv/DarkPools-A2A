\section{IfThenElse}

A basic syntax in most programming languages is the \textit{if} statement.
However, in an MPC protocol, this can create challenges if the
condition of the statement is a secret.
The protocol cannot reveal which branch is executed, and so must
execute both branches.

To provide something akin to the traditional \textit{if} statement,
without significant overhead,
the Silent Compute platform implements the \IfThenElseZ~protocol.
This allows a variable to be set according to a secret condition,
represented as a bit.
This has the same effect as a multiplexer, where the
condition acts as the select bit.
(Note that if the condition is \emph{true}, represented by \bit{1},
the first item is selected.)

\begin{protocol}

	\IfThenElse{\bit{c}}{\bit{x}}{\bit{y}}:\\
	\indent $\bit{z} = \bit{x} \oplus \bit{y} $ \\
	\indent $\bit{d} = \bit{z} \wedge \bit{c} $ \\	
	\indent $\bit{r} = \bit{d} \oplus \bit{y} $\\

\end{protocol}

Proof of correctness:\\
If $c = 0$, then $d = 0$ and $r = y$ as required.
If $c = 1$, then $d = z = x \oplus y$ and $r = x \oplus y \oplus y = x$ as required.


