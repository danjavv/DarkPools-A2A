\section{Boolean Share Representation}

We adopt the $2$-out-of-$3$ secret sharing scheme from \cite{EC:FLNW17}, which is defined as follows.
To share a bit $x$, three random bits $s_1, s_2, s_3 \in \{0, 1\}$ are selected such that the constraint $s_1 \oplus s_2 \oplus s_3 = x$ is satisfied. 
Then, the shares are defined as follows:

\begin{itemize}
    \item $P_1$'s share is $(t_1, s_1)$, where $t_1 = s_3 \oplus s_1$.
    \item $P_2$'s share is $(t_2, s_2)$, where $t_2 = s_1 \oplus s_2$.
    \item $P_3$'s share is $(t_3, s_3)$, where $t_3 = s_2 \oplus s_3$.
\end{itemize}

It is evident that no individual share reveals any information about the value of $x$.
Moreover, any two parties can jointly reconstruct $x$. For instance, by combining $(t_1, s_1)$ and $(t_2, s_2)$ one can compute $x = t_1 \oplus s_2$. 
We denote the Boolean RSS sharing of $x$ using this scheme as $\bit{x}$. 
This secret sharing, when used in the ABB functionality, ensures that the protocol retains the same properties as if it were directly interacting with the functionality.