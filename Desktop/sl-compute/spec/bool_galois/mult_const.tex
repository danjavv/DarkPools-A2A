\section{Multiplication by a Public Constant}

Since parties hold (replicated) additive shares of secrets, 
multiplying by a public constant is simple and requires no communication.
This assumes that all parties locally have the same definition
for multiplication in the Galois field \GFtwol,
which includes that for any $l$ they use the same irreducible polynomial.

\begin{protocol}[Galois Multiplication by a Public Constant]
	\GalMultConst{\galois{x}}{c}:
		\begin{enumerate}
			\item Let $\private{(x_i, x_{i+1})}{i}$
				be the share of $x$ held by $P_i$.
			\item $P_i$ computes $z_i = x_i \Gmult c$ and
				$z_{i+1} = x_{i+1} \Gmult c$.
			\item $P_i$ sets its shares of the result to be
				$\private{(z_i, z_{i+1})}{i}$.
			\item The parties generate a unique consistent
				public identifier for the result.
		\end{enumerate}
\end{protocol}

\textbf{Correctness:}
The argument for correctness is identical to that of the addition protocol.
If the protocol has correctly implemented the ABB up to this point,
the sharing of the honest parties is consistent and correct.
The honest parties will therefore have a correct and consistent
sharing of the output.
