\section{Initialization and Randomness Generation}

It is possible, with the correct setup, to generate 
Replicated Secret Sharing (RSS) shares
of pseudorandom values without interaction \cite{TCC:CraDamIsh05}.
Our Boolean Bit ABB implements this in the context of a sharing
of Boolean values, but the technique is equally applicable
to RSS in a Galois Field.
This means that our Galois ABB is able to generate new random values,
which are stored in the ABB, without any communication between parties.

Specifically, the setup causes each pair of parties to hold a PRF key
which is known to them, but unknown to the third party.
If each pair uses this PRF to create a new pseudorandom value,
the values create correspond to a RSS of a random value.

\begin{protocol}[Setup]
	\GalSetup{}:\\
	\textbf{Auxilary Input}: 
	Each party holds a security parameter $\kappa$,
	and a pseudorandom function 
	$F:\{0,1\}^{\kappa} \times \{0,1\}^\kappa \rightarrow \{0,1\}^\kappa$
	\begin{enumerate}
		\item $P_i$ picks a random key $k_i \in \{0, 1\}^\kappa$.
		\item $P_i$ sends $k_i$ to $P_{i+1}$.
		\item $P_i$ stores $(k_{i-1}, k_i)$.
		\item Each $P_i$ stores a static variable $id$ and assigns $id=0$.
	\end{enumerate}
\end{protocol}

\begin{protocol}[Random]
	\GalRand{l}:
	\begin{enumerate}
		\item Let $t = \lceil l/\kappa \rceil$
		\item Each $P_i$ 
			set $x_i$ to the first $l$ bits of
			$F_{k_i}(id) || \ldots || F_{k_i}(id+t-1)$.\\
			(In the common case, $t=1$, this is simply
			the first $l$ bits of $F_{k_i}(id)$.)
		\item Each $P_i$ 
			set $x_{i-1}$ to the first $l$ bits of
			$F_{k_{i-1}}(id) || \ldots || F_{k_{i-1}}(id+t-1)$.
		\item Each $P_i$ sets $y_i = \BoolArrToGal{x_i}$ and\\ 
			$y_{i-1} = \BoolArrToGal{x_{i-1}}$.
		\item Set $id = id + t$.
		\item Each $P_i$ sets its share as $(y_{i-1}, y_i)$. 
	\end{enumerate}
\end{protocol}
