\section{Inner Product With Error}

The technique for multiplication can easily be extended to computing 
the inner product (or dot product) of two vectors of Galois field elements.
Observe that in the multiplication protocol, 
the parties first created an additive sharing of the product by multiplying local shares.
The inner product protocol first applies this method to obtain additive 
sharings of all of the products 
$(\vec{X}_0 \Gmult \vec{Y}_0), \ldots, (\vec{X}_{n-1} \Gmult \vec{Y}_{n-1})$.
Additive sharings can be summed by each party simply adding their shares locally.
This results in an additive sharing of the inner product,
which is re-shared to a Replicated Secret Sharing by each party
inputting their additive share.
Note that even though $n$ multiplications occur, only 1 re-sharing is necessary.
As such the communication cost of an inner product is identical
to that of a multiplication.
Furthermore, the adversary's ability to distort the result is the same
as with the multiplication:
they can choose to input an incorrect value during the re-sharing,
but this will only have the effect of introducing an additive error 
to the result.

\begin{protocol}[Galois Inner Product with Error]
	\GalInnerProdWithError{\gvec{X}}{\gvec{Y}}:
	\begin{enumerate}
		\item Let $\{\private{(\vec{X}_{i,t}, \vec{X}_{i+1,t})}{i}\}_{t=0,\ldots,n-1}$ and
			$\{\private{(\vec{Y}_{i,t}, \vec{Y}_{i+1,t})}{i}\}_{t=0,\ldots,n-1}$
			represent $P_i$'s shares of $\vec{X}$ and $\vec{Y}$
			respectively.
		\item $P_i$ locally computes:\\ 
			$\vec{R}_{i,t} = (\vec{X}_{i,t} \Gmult \vec{Y}_{i,t}) 
				\Gadd (\vec{X}_{i,t} \Gmult \vec{Y}_{i+1,t})
				\Gadd (\vec{X}_{i+1,t} \Gmult \vec{Y}_{i,t})$ for $t=0,\ldots,n-1$.\\
			$r_i = \vec{R}_{i,0} \Gadd \ldots \Gadd \vec{R}_{i,n-1}$
		\item For each $i \in \{1,2,3\}$ (in parallel) set\\
				$\galois{r_i} = \InputGalFrom{\private{r_i}{i}}{i}$
		\item Compute $\galois{z} = \galois{r_1} \Gadd \galois{r_2} \Gadd
			\galois{r_3}$
		\item Return $\galois{z}$.
	\end{enumerate}
\end{protocol}

The proof of security is similar to that for multiplication, so we omit it.
