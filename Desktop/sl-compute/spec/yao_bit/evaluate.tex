\section{Evaluation}

This operation is carried out by the evaluator during the online phase to compute the circuit’s output. As with the garbling procedure, the underlying technique is based on the half-gates construction from~\cite{EC:ZahRosEva15}, while the protocol’s representation follows the approach described in~\cite{EPRINT:GYWYL23}. This phase assumes that input garbling and label sharing have already been completed, and it produces the wire labels corresponding to the output gates of the circuit. Notably, unlike the cited protocol representations, this protocol has the evaluator output the full wire labels rather than only their \textsf{lsb}, enabling later authentication and allowing the outputs to be represented as Yao-bit sharings, as described in Section~\ref{sec:yaobit:repr}.

\begin{protocol}[Evaluate]	
	$\Evaluate{\vec{F}}{\vec{X}}$:\\
	\indent  for $i \in \textsf{Inputs}(\vec{F})$: \\
	\indent \indent  $W_i = \vec{X}_i$ \\
	\indent  for $i \notin \textsf{Inputs}(\vec{F})$ \{\textit{ in topo. order}\}: \\
	\indent \indent  $\{a, b\} = \textsf{GateInputs}(\vec{F}, i), \{c\} = \textsf{GateOutputs}(f, i)$ \\
	\indent \indent  if $i \in \textsf{XorGates}(\vec{F})$: \\
	\indent \indent \indent  $W_i = W_a \oplus W_b$ \\
	\indent \indent  else if $i \in \textsf{AndGates}(f)$:\\
	\indent \indent \indent  $s_a = \textsf{lsb}(W_a)$ \\
	\indent \indent \indent  $s_b = \textsf{lsb}(W_b)$ \\
	\indent \indent \indent $k_i^0 = 2 \cdot \textsf{GateIndex}(f, i) - 1$\\
	\indent \indent \indent $k_i^1 = 2 \cdot \textsf{GateIndex}(f, i)$ \\
	\indent \indent \indent  $(G_i^0, G_i^1) = \vec{F}_i$ \\
%	\indent \indent \indent  $W_{G_i} = H(W_a, k_i^0) \oplus s_a TG_i$ \\
%	\indent \indent \indent  $W_{E_i} = H(W_b, k_i^1) \oplus s_b (TE_i \oplus W_a)$ \\
	\indent \indent \indent  $W_i = H(W_a, k_i^0) \oplus H(W_b, k_i^1) \oplus s_a G_i^0 \oplus s_b (G_i^1 \oplus W_a)$ \\
	\indent  for $i \in \textsf{Outputs}(\vec{F})$:\\
	\indent \indent  $Y_i = W_i$ \\
	\indent  return $\vec{Y}$
\end{protocol}