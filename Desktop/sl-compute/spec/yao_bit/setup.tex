\section{Setup}
\label{sec:yaobit:setup}

This operation is performed during the offline phase for each instantiation of the garbled circuit. The initial steps involving key exchanges can be executed once per session, allowing the same keys to be reused throughout the session.

As in previous descriptions, we assume that $P_1$ and $P_2$ act as garblers, while $P_3$ serves as the evaluator. Another assumption we make is that the input evaluator provides during the protocol is known beforehand, and the circuit is appropriately chosen such that the input operation is supported. To be specific, if in  We also assume that the evaluator’s input is known in advance, and that the circuit is selected accordingly to support the input-sharing mechanism. Specifically, if $f(x_1, x_2, x_3)$ is the function to be securely evaluated and $x_3$ is the evaluator’s input known beforehand, then the setup is performed for the modified circuit $f'(x_1, x_2, x_3, x_4) = f(x_1, x_2, x_3 \oplus x_4)$.

\begin{protocol}[Setup]
	$\SetupYao{f}{1^k}$: \\
	\indent \textcolor{blue}{\textit{$P_3$ does the following:}} \\
	\indent $crs \gets_r \{0,1\}$ \\
	\indent Send $crs$ to $P_1$ and $P_2$ \\
	\indent \textcolor{blue}{\textit{$P_1$ does the following:}} \\
	\indent $r \gets \{0,1\}^k$ \\
	\indent Send $r$ to $P_2$ \\
	\indent \textcolor{blue}{\textit{$P_1$ and $P_2$ do the following:}} \\
	\indent $P_1$ and $P_2$ instantiate $PRF$ using $r$ to extract randomness. \\
	\indent $(\vec{F}, \vec{e}, \vec{d}) = \Garble{1^k}{f}$ \\
	\indent Send $\vec{F}$ to $P_3$ \\
	\indent return $crs, r, \vec{e}, \vec{d}$\\
	\indent \textcolor{blue}{\textit{$P_3$ does the following:}} \\
	\indent Check if same $\vec{F}$ is sent by $P_1$ and $P_2$. If not abort. \\
	\indent return $crs, \vec{F}$
\end{protocol}