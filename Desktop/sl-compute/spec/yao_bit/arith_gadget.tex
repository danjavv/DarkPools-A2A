\section{Arithmetic Garbling Gadget}

This section describes a gadget to garble a function $f: \{0,1\}^\ell \rightarrow \Zp$.  Specifically, the function is defined as follows, where $f(x) = \langle \vec{U}, \vec{X} \rangle$, where $\vec{U} \in (\Z_q^*)^\ell$ and $\vec{X} \in \{0,1\}^\ell$. This gadget is intended to be plugged into the Yao-bit functionality, to export the output as an arithmetic value instead of a binary value. Since the output of this gadget is an arithmetic value, this gadget cannot be plugged into a binary garbled circuit protocol. To instantiate the gadget, we adopt the protocol from Section 6 of \cite{C:GKMN21}. Their protocol works in the privacy free setting. However, since we require privacy, we adopt the point-and-permute technique for binary garbled circuits to add privacy to the gadget. All algorithms make use of the key derivation function
KDF. 

\begin{protocol}[Arithmetic Garbling Gadget]
	$\GarbleYaotoArith{\{W_i\}_{i \in [\ell]}}{\Delta}{q}:$ \\
	\indent $a \gets_r \Z_q^*$ \\
	\indent for $i \in [\ell]:$ \\
	\indent \indent $p_i = \mathsf{lsb}(W_i)$ \\
	\indent \indent $b_i = -\mathsf{KDF}(i, W_i) + p_i \cdot \vec{U}_i \cdot a$ \\
	\indent \indent $\vec{C}_i = \mathsf{KDF}(i, W_i \oplus \Delta) + b_i  + (1 - p_i) \cdot \vec{U}_i \cdot a$ \\
	\indent $b = \sum_{i \in [\ell]}b_i$ \\
	\indent $\mathsf{de} = (a, b)$ \\
	\indent return $\vec{C}, \mathsf{de}$ \\
	
	\noindent 
	$\EvaluateYaotoArith{\vec{C}}{\vec{X}}:$ \\
	\indent for $i \in [\ell]$: \\
	\indent \indent $\lambda_i = \mathsf{lsb}(\vec{X}_i)$ \\
	\indent \indent $\vec{Z}_i = \lambda_i \cdot \vec{C}_i - \mathsf{KDF}(i, \vec{X}_i)$ \\
	\indent $z = \sum_{i \in [\ell]}\vec{Z}_i$ \\
	\indent return $z$ \\
	
	\noindent 
	$\OutputYaotoArith{\mathsf{de}}{z}$: \\
	\indent $(a, b) = \mathsf{de}$ \\
	\indent $o = a^{-1} \cdot (z - b)$ \\
	\indent return $o$
\end{protocol}

\paragraph{\bf Proof of Correctness:} Let the input wire labels correspond to $x$, where the bit decomposition of $x$ can be defined as $x = \sum_{i=0}^{\ell - 1}x_i \cdot 2^i$. Now, from the point-and-permute technique, we know that $\lambda_i = p_i \oplus x_i$. 