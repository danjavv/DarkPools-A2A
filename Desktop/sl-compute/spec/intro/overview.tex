\section{\SLCompute~Overview}

The \SLCompute~platform allows the secure outsourcing of computation to a 
distributed platform, behaving as a cryptographically secure virtual machine.
The initial platform is designed for the following security setting:
\begin{itemize}
	\item There are 3 compute parties.
	\item At most one of these parties is corrupted.
	\item The corrupted party may be \emph{malicious}:
		it may behave arbitrarily but doing so cannot allow it to learn more
		information or cause the protocol to produce incorrect outputs.
	\item The protocol is \emph{secure with abort}: the corrupted party may cause the
		protocol to stop without completing.
	\item Aborts are \emph{not identifiable}: it is often not clear which party
		caused the protocol to abort.
	\item Aborts are \emph{not fair}: the corrupted party may learn an output and cause
		an abort before other parties learn the output.
	\item Communication is \emph{synchronous}: there is a publicly-known upper-bound
		on message latency. 
	\item The platform is \emph{computationally secure}:
		that is there is a negligible probability of incorrect outputs
		and negligible statistical leakage of additional information,
		against a computationally bounded adversary.%
		\footnote{
			In general, we assume Pseudo-Random Functions.
			Any further computational assumptions will be stated explicitly.
		}
\end{itemize}

Note that the synchrony assumption means that a party that goes offline
(or has a latency above the publicly-known upper-bound)
will be treated the same as a malicious party.
However, since the protocol is secure with abort,
this is simply a denial of service, which is an expected consequence of an offline party.
If a party observes incorrect behaviour, there is a unique message $\bot$
which it can send to other parties, causing them to abort the protocol.

The first instantiation of the platform supports the following functionalities:
\begin{enumerate}
	\item Standard basic operations on bits and positive integers.
	\item Basic analytic queries.
		It allows sums, counts and averages to be computed
		over records that satisfy specified equality/inequality-based conditions.
	\item The establishment a secure (encrypted and authenticated) 
		connection with an API end-point, including decrypting to secret-shares.
\end{enumerate}

The \SLCompute~platform can be extended in the future.
The security setting can be extended, for instance by extending to the 2-party setting
or by providing identifiable abort.
The functionality can also be extended, for instance by allowing more sophisticated
queries or analysis based on machine learning techniques.
However, this document currently is restricted to providing a specification that
matches the security setting and functionalities detailed above.
