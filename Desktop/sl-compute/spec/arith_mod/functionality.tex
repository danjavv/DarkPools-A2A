\section{Modular Arithmetic ABB Functionality}

The Modular Arithmetic ABB allows the inputting and outputting of elements of the field $\Zp$ (where $p$ is a prime), and basic operations on these elements.
Modular arithmetic over $\Zp$ is useful in a wide range of applications. 
It enables efficient instantiation of arithmetic for a fixed prime $p$, which is central to many cryptographic constructions such as elliptic curves.
It also supports secure computation via arithmetic secret sharing, allowing the use of MACs for authentication of shared values.
In order to make explicit the distinction between operations in the field $\Zp$, we denote addition and multiplication using $\MAadd$ and $\MAmult$, respectively.
This is summarized in the functionality definition below.
More detailed descriptions, as well as the protocols implementing each of these functionalities, will be presented in later sections.

\begin{functionality}
	
	\InputModArith{x}: Given a public integer $x$ in $\Zp$, store it in the ABB and return a public identifier to the variable. 
	(Shorthand: $\modarith{x}$.)\\
	
	\InputModArithFrom{x}{i}:  Given a private value, $\private{x}{i}$, held by $P_i$, where $x \in \Zp$,
	store $x$ in the ABB and return a public identifier to the variable. 
    Note that $P_i$'s local \emph{identifier} for $\private{x}{i}$ can also be private to $P_i$. \\
	
	\OutputModArith{\modarith{x}}: Given an element of $\Zp$ which is stored in the ABB under identifier $x$, reveal it to all parties. \\
	
	\OutputModArithTo{\modarith{x}}{i}:  Given an element of $\Zp$ which is stored in the ABB under identifier $x$, reveal it to $P_i$. \\
	
	\ModArithAdd{\modarith{x}}{\modarith{y}}: Given $x$ and $y$, both elements of $\Zp$ stored in the ABB, return \modarith{x \MAadd y}. (Shorthand: $\modarith{x} \MAadd \modarith{y}$.) \\

    \ModArithMultConst{\modarith{x}}{c}: Given $x$ and $c$, both elements of $\Zp$, where $x$ is stored in the ABB and $c$ is public, return $\modarith{x \MAmult c}$. \\

\end{functionality}
